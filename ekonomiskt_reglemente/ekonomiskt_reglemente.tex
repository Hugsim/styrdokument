%LaTeX-inställningar%%%%%%%%%%%%%%%%%%%%%%%%%%%%%%%%%%%%%%%%%%%%%%%%%%%
% Kompliera med pdflatex!!!
\documentclass[a4paper, 10pt]{article}

\usepackage{hyperref}
\usepackage[utf8]{inputenc}
\usepackage[T1]{fontenc}
\usepackage[swedish]{babel}
\usepackage{graphicx}
%\usepackage{ae} %För riktiga fonts?
\usepackage{fancyhdr}
\usepackage{lastpage}
\topmargin -20.0pt
\headheight 56.0pt
\setcounter{secnumdepth}{5}

%FYLL I VID ÄNDRINGAR%%%%%%%%%%%%%%%%%%%%%%%%%%%%%%%%%%%%%%%%%%%%%%%%%%
\newcommand{\updated}{2019-02-12} %Insert




%Dokumentstart%%%%%%%%%%%%%%%%%%%%%%%%%%%%%%%%%%%%%%%%%%%%%%%%%%%%%%%%%
\begin{document}
\pagestyle{fancy}

%Header%%%%%%%%%%%%%%%%%%%%%%%%%%%%%%%%%%%%%%%%%%%%%%%%%%%%%%%%%%%%%%%%
\renewcommand{\headrule}{\vbox to 0pt{\hfill\hbox to 292pt{\hrulefill}}}
\lhead{
\raisebox{-25pt}[0pt][10pt]{\includegraphics[width=60pt]{dteklogo.pdf}}
\parbox[b]{200pt}{
\textbf{Datateknologsektionen}\\
Chalmers studentkår\\
Ekonomiskt reglemente}}
\rhead{ \flushright
Sidan \thepage\ av \pageref{LastPage}\\
Uppdaterad \updated}

%Footer%%%%%%%%%%%%%%%%%%%%%%%%%%%%%%%%%%%%%%%%%%%%%%%%%%%%%%%%%%%%%%%%
\renewcommand{\footrulewidth}{\headrulewidth}
\lfoot{\flushleft Datateknologsektionen\\
    Rännvägen 8\\
    412 58 Göteborg}
\cfoot{}
\rfoot{ \flushright styret@dtek.se\\
www.dtek.se}
\newpage

%Titlepage%%%%%%%%%%%%%%%%%%%%%%%%%%%%%%%%%%%%%%%%%%%%%%%%%%%%%%%%%%%%%
\vspace*{\fill}
\begin{center}
{\Huge \textbf{Ekonomiskt reglemente för Datateknologsektionen}}\\
\includegraphics[width=300pt]{dteklogo.pdf}
\\{\LARGE Chalmers, Göteborg}
\end{center}
\vspace*{\fill}
\begin{center}
{\LARGE Uppdaterad: \updated}
\end{center}
\vspace*{\fill}


\newpage
\setcounter{tocdepth}{1}
\tableofcontents
\newpage

%Allmänt%%%%%%%%%%%%%%%%%%%%%%%%%%%%%%%%%%%%%%%%%%%%%%%%%%%%%%%%%%%%%%%
\section{Allmänt}
\subsection{}
Detta reglemente har skapats för att skapa tydlighet över vad sektionens pengar får och inte får användas till. Det är därmed tänkt att vara ett hjälpmedel för styrelsen, revisorerna och sektionens kommittéer i deras arbete. Grundfilosofin är att sektionens pengar är till för sektionens samtliga medlemmar, vilket i praktiken innebär att arrangemang som inte är öppna för dessa inte får belasta sektionens ekonomi. Viktigt att tänka på vid tolkandet utav reglementet är att det inte har utformats för att i onödan begränsa kommittéers möjligheter eller lust till roliga arrangemang och upptåg.
\subsection{Prisbasbelopp}
\begin{itemize}
  \item Det prisbasbelopp som skall användas under hela verksamhetsåret är det prisbasbelopp som är aktuellt vid verksamhetsårets början.
  \item Delar av prisbasbelopp skall avrundas till närmaste hundratal. 
\end{itemize}

%D-Styret%%%%%%%%%%%%%%%%%%%%%%%%%%%%%%%%%%%%%%%%%%%%%%%%%%%%%%%%%%%%%%
\section{D-Styret}
\subsection{}
D-Styret är ansvariga för sektionens hela ekonomi
\subsection{}
Ordförande är skyldig att
\begin{itemize}
  \item Teckna sektionens firma
  \item Se till att ordförande i varje kommitté har tillgång till och kunskap om sektionens stadgar, reglemente och ekonomiska reglemente 
\end{itemize}
\subsection{}
Kassören är skyldig att
\begin{itemize}
  \item Teckna sektionens firma
  \item Se till att kassör i varje kommitté har tillgång till och kunskap om sektionens stadgar, reglemente och ekonomiska reglemente 
  \item Fortlöpande kontrollera sektionens räkenskaper och bokföring
  \item I samråd med styrelsen upprätta budgetförslag till första ordinarie höstmötet
  \item Till varje sektionsmöte kunna redogöra för sektionens ekonomiska ställning
  \item Utbilda nya sektionsfunktionärer i hur sektionens bokförings och redovisningssystem skall användas
  \item Ansvara för styrelsens handkassa
\end{itemize}
\subsection{}
Styret har rättighet att inom ramen för styrets budget representera sektionen och ordna förtäring till styrets ledamöter i samband med styrelsemöten. Denna förtäring skall vara med som post i styrelsens budget.

%Sektionskommittéer%%%%%%%%%%%%%%%%%%%%%%%%%%%%%%%%%%%%%%%%%%%%%%%%%%%%
\section{Sektionskommittéer}
\label{sec:sektionsforeningar}
\subsection{}
Ordförande i sektionskommitté är skyldig att
\begin{itemize}
\item Kontinuerligt meddela kommitténs ekonomiska status till styret
\item Tillsammans med kommitténs kassör ansvara för att kommittén förvaltar sina tillgångar i enlighet med sektionens stadgar, reglementen och beslut.
\item Tillsammans med kommitténs kassör skriva ett bindande avtal med
styret angående skuldfrågan vid felaktig bokföring
\item Tillsammans med kommitténs kassör teckna kommitténs firma
\end{itemize}
\subsection{}
Kassör i sektionskommitté är skyldig att
\begin{itemize}
\item Föra kassabok av sådan typ som godkänts av sektionens revisorer
\item Tillsammans med kommitténs ordförande ansvara för att kommittén förvaltar sina tillgångar i enlighet med sektionens stadgar, reglementen och beslut.
\item Tillsammans med kommitténs ordförande skriva ett förbindande avtal med Styret angående skuldfrågan vid felaktigt förd bokföring
\item På varje sektionsmöte redovisa kommitténs ekonomiska situation
\item Arkivera sektionens bokföringar, på en plats anvisad av Styret, så lång tid som föreskrivs för den organisationsform som datateknologsektionen är.
\item Tillsammans med kommitténs ordförande teckna kommitténs firma
\end{itemize}
\subsection{}
Varje kommitté skall påbörja en ny kassabok i starten av sitt verksamhetsår
\subsection{Äskning}
Kommittéer som önskar extra ekonomiska medel till sin verksamhet skall inkomma med önskemål och skäl därför till styret. Styret kan bevilja extra medel om beloppet understiger 0,25 prisbasbelopp, om beloppet är större måste först sektionsmöte tillfrågas.
\subsection{Kapital}
\label{sec:sektionsforeningar_startkapital}
\subsubsection{Startkapital}
DRust, DAG, Delta, D6 och DNollK skall vid mandatperiodens början ha 0 prisbasbelopp
i tillgångar. DRust, Delta, D6 och DNollk har rätt att från styrelsen låna
upp till 0.25 prisbasbelopp, DAG har rätt att låna upp mot 1.25 prisbasbelopp.
\subsubsection{Slutkapital}
DRusts, DAGs, Deltas, D6s och DNollKs tillgångar samt skulder skall vid slutet av mandatperioden tillfalla styrelsen.
\subsubsection{Stödkapital}
Styret finner sig rätt att temporärt låna ut
pengar till en kommitté för att betala kritiska skulder vid inväntan av inkommande
kapital.
\subsection{Budget}
För arrangemang med förväntad omsättning överstigande 0.75 prisbasbelopp skall en noggrann budget upprättas och godkännas av styret innan arrangemanget får genomföras.
\subsection{Arbetskläder}
Kommitté får lägga upp till 0.025 prisbasbelopp per person på arbetskläder. Rimliga arbetskläder kan anses vara en overall eller motsvarande, t-shirt samt annan
tröja/piké.
\subsection{Jobbarmat}
Vid arrangemang i kommitténs verksamhet får kommitténs ekonomi belastas med utgifter för inköp av jobbarmat enligt följande tidsåtgång för arrangemanget:
\begin{itemize}
    \item 1-4 timmar: 30 kr/per person (ca 0.00075 prisbasbelopp)
    \item 4-8 timmar: 60 kr/per person (ca 0.0015 prisbasbelopp)
    \item Mer än 8 timmar: 90 kr/per person (ca 0.00225 prisbasbelopp)
\end{itemize}
Dessa utgifter får ej överskrida kommitténs ekonomiska kapacitet. Jobbarmatens tillagning får inte medräknas i arrangemangets tidsåtgång

%Hobbykommittéer och andra grupper under sektionen%%%%%%%%%%%%%%%%%%
\section{Hobbykommittéer och andra grupper under sektionen}
\subsection{}
Ekonomiska medel för hobbykommittéer och andra grupper under sektionen skall regleras av styrets budget
\subsection{}
I de fall en hobbykommitté eller annan grupp under sektionen har beviljats egen ekonomi gäller även §\ref{sec:sektionsforeningar}, med undantag för §\ref{sec:sektionsforeningar_startkapital}, för dessa.

%Internrepresentation%%%%%%%%%%%%%%%%%%%%%%%%%%%%%%%%%%%%%%%%%%%%%%%%%%
\section{Internrepresentation}
\subsection{}
\label{sec:internreps}
Varje sektionskommitté och DNS har rättighet att varje verksamhetsår bekosta följande på sind ekonomi:
\begin{itemize}
    \item[-] En eller flera teambuildingaktiviteter för upp till 0,015 prisbasbelopp per person sammanlagt.
    \item[-] Överlämning för upp till 0,025 prisbasbelopp.
    \item[-] Aspning och aspplagg för upp till 0,065 prisbasbelopp sammanlagt.
\end{itemize}

Ovanstående får inte överskrida kommitténs ekonomiska resurser. Alkoholhaltiga drycker får ej bekostas med kommitténs pengar vid ovanstående arrangemang.

\subsection{}
Styrelsen har rätt att bevilja hobbykommittéer eller andra grupperingar under teknologsektionen rättighet motsvarande §\ref{sec:internreps} om det anses vara motiverat.

%Sponsring%%%%%%%%%%%%%%%%%%%%%%%%%%%%%%%%%%%%%%%%%%%%%%%%%%%%%%%%%%%%%
\section{Sponsring}
\subsection{}
Kommittéer får inte söka sponsring utan samråd med datas
arbetsmarknadsgrupp, DAG.

%Fonder%%%%%%%%%%%%%%%%%%%%%%%%%%%%%%%%%%%%%%%%%%%%%%%%%%%%%%%%%%%%%%%%
\section{Fonder}
\subsection{Kapitalfonden}
\subsubsection{Syfte}
\label{sec:kapitalfond_syfte}
Syftet med kapitalfonden är att avlasta sektionens respektive de olika sektionskommittéernas ekonomi från stora investeringar. Pengarna skall användas till saker som har ett bestående värde, samt är till gagn för sektionens medlemmar direkt eller indirekt. Pengar skall ej användas till driftbidrag eller stöd för förgänglig verksamhet. Fonden skall ej användas till verksamhet som lokalfond är ämnad för.
\subsubsection{Förvaltning}
Fonden förvaltas av sektionens styrelse.
\subsubsection{Avsättning}
\begin{itemize}
\item En av styrelsen budgeterad summa som godkänts av sektionsmötet
\item All avkastning ifrån kapitalfonden tillförs kapitalfonden.
\end{itemize}
\subsubsection{Uttag}
\begin{itemize}
\item Sektionsstyrelsen har rätt att bevilja uttag ur fonden på belopp upp till totalt 0,25 prisbasbelopp per år. Uttag av belopp överstigande 0,25 prisbasbelopp skall godkännas av sektionsmöte innan medel utbetalas. Varje uttag skall redovisas inför nästkommande sektionsmöte.
\item Sektionsmötet äger rätt att vid välmotiverat behov göra uttag i strid med syftet i §\ref{sec:kapitalfond_syfte}
\end{itemize}
\subsubsection{Revision}
Revidering sker varje år av sektionens revisorer. Revision sammanfaller med övrig på sektionen förekommande.
\subsection{Lokalfonden}
\subsubsection{Syfte}
\label{sec:lokalfond_syfte}
Syftet med lokalfonden är att säkra medel för underhåll och reparationer av sektionens lokaler, samt för inköp av inventarier. Pengarna skall användas till större reparationer och ommålningar av sektionslokalerna, samt möbler till trivselytor där teknologen i gemen har tillträde.
\subsubsection{Förvaltning}
Fonden förvaltas av sektionens styrelse.
\subsubsection{Avsättning}
\label{sec:lokalfond_avsattning}
\begin{itemize}
\item Tio (10) \% av under verksamhetsåret influtna sektionsavgifter tillförs lokalfonden.
\item En av styrelsen budgeterad summa som godkänts av sektionsmötet
\item All avkastning ifrån lokalfonden tillförs lokalfonden.
\end{itemize}
\subsubsection{Uttag}
\begin{itemize}
\item Sektionsstyrelsen disponerar 3/4 av årets tillförda kapital,
enligt första punkten i §\ref{sec:lokalfond_avsattning}, för basdrift av sektionslokalerna. Sektionsstyrelsen har rätt att besluta om uttag på överstigande belopp, dock skall detta redovisas inför nästkommande sektionsmöte. Vid detta sektionsmöte skall i sådana fall även genomförda eller planerade inköp, reparationer och underhåll redovisas.
\item Sektionsmötet äger rätt att vid välmotiverat behov göra uttag i strid med syftet i §\ref{sec:lokalfond_syfte}.
\end{itemize}
\subsubsection{Revision}
Revidering sker varje år av sektionens revisorer. Revision
sammanfaller med övrig på sektionen förekommande.
\end{document}

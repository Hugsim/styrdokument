\section{Sektionsstyrelsen}
\subsection{Befogenheter}
Sektionsstyrelsen handhar i överensstämmelse med denna stadga, befintligt reglemente, befintligt ekonomiskt reglemente samt av sektionsmötet fattade beslut den verkställande ledningen av sektionens verksamhet.
\subsection{Sammansättning}
Sektionsstyrelsen består av:
\begin{itemize}
\item Ordförande
\item Vice ordförande
\item Kassör
\item Sekreterare
\item i reglementet fastställda medlemmar
\end{itemize}
Ordförande och kassör i sektionsstyrelsen skall vara myndiga.
\subsection{Rättigheter}
Sektionsstyrelsen äger rätt att i namn och emblem använda teknologsektionens namn och dess symboler i enlighet med Chalmers Studentkårs policies.
\subsection{Ansvarighet}
Sektionsstyrelsen ansvarar inför sektionsmötet för teknologsektionens verksamhet och ekonomi.
\subsection{Firmateckning}
Ordförande i sektionsstyrelsen samt dess kassör tecknar teknologsektionens firma var för sig.
\subsection{Styrelsemöte}
Sektionsstyrelsen sammanträder minst tre gånger per läsperiod.
\subsection{Utlysande}
\subsubsection{}
Sektionsstyrelsen sammanträder på kallelse av ordförande eller vice ordförande i sektionsstyrelsen.
\subsubsection{}
Medlem av sektionsstyrelsen äger rätt att hos vice ordförande i sektionsstyrelsen begära utlysande av styrelsemöte. Sådant möte skall hållas inom 5 läsdagar.
\subsection{Beslutförhet}
Sektionsstyrelsen är beslutsmässigt när minst 50\% av medlemmarna är närvarande. Ordförande eller vice ordförande skall närvara.
\subsection{Överklagande}
Beslut av sektionsstyrelsen som strider mot kårens eller teknologsektionens stadga, reglemente, ekonomiska reglemente samt policy får undanröjas av kårens fullmäktige. Sådant beslut skall tas upp till prövning om det begärs av en kårmedlem då det rör kårens stadga, eller av teknologsektionsmedlem då det rör sektionens stadga, reglemente, ekonomiska reglemente eller policy.
\subsection{Protokoll}
Protokoll skall föras vid styrelsemöte, justeras av två medlemmar av sektionsstyrelsen och anslås på teknologsektionens anslagstavla senast fem läsdagar efter mötet.
\subsection{Avsättning}
\subsubsection{}
För att avsätta sektionsstyrelsen krävs att ärendet är anslaget senast tre läsdagar innan sektionsmöte, och minst 2/3 av de röstberättigade vid mötet är om beslutet ense.
\subsubsection{}
Vid detta möte skall interimsstyrelse och ny valberedning väljas. Interimsstyrelsen utfärdar kallelse till extra sektionsmöte där ny ordinarie styrelse skall väljas. Detta sektionsmöte skall hållas inom 15 läsdagar från det sektionsmöte då interimsstyrelsen valdes och under ordinarie terminstid.
\subsubsection{}
Interimsstyrelsen övertar ordinarie sektionsstyrelsens befogenheter och skyldigheter tills dess en ny ordinarie sektionsstyrelse är vald.

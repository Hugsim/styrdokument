% LaTeX-inställningar%%%%%%%%%%%%%%%%%%%%%%%%%%%%%%%%%%%%%%%%%%%%%%%%%%
% Kräver dtek-tex (github.com/dtekcth/dtek-tex)
% KOMPILERAS MED xelatex!
\documentclass[a4paper]{dtek}
\setcounter{secnumdepth}{5}
\title{Stadgar3}


%FYLL I VID ÄNDRINGAR%%%%%%%%%%%%%%%%%%%%%%%%%%%%%%%%%%%%%%%%%%%%%%%%%%
\newcommand{\updated}{2020-11-21} %Insert

%Dokumentstart%%%%%%%%%%%%%%%%%%%%%%%%%%%%%%%%%%%%%%%%%%%%%%%%%%%%%%%%%
\begin{document}
\makeheadfoot

%Titlepage%%%%%%%%%%%%%%%%%%%%%%%%%%%%%%%%%%%%%%%%%%%%%%%%%%%%%%%%%%%%%
\vspace*{\fill}
\begin{center}
{\Huge \textbf{Stadgar för Datateknologsektionen}}
\par\bigskip
\includegraphics[width=300pt]{dteklogo.pdf}
\par\bigskip
{\LARGE Chalmers, Göteborg}
\end{center}
\vspace*{\fill}
\begin{center}
{\LARGE Uppdaterad: \updated}
\end{center}
\vspace*{\fill}


\newpage
\setcounter{tocdepth}{1}
\tableofcontents
\newpage

%Allmänt%%%%%%%%%%%%%%%%%%%%%%%%%%%%%%%%%%%%%%%%%%%%%%%%%%%%%%%%%
\section{Allmänt}
\subsection{Ändamål}
\subsubsection{}
Datateknologsektionen vid Chalmers, härmed benämnd teknologsektionen, är en ideell förening bestående av studerande vid utbildningsprogrammet för datateknik vid Chalmers.
\subsubsection{}
Teknologsektionen har till uppgift att verka för sammanhållning mellan medlemmarna och tillvarata deras gemensamma intressen i utbildnings- och studiesociala frågor.
\subsubsection{}
Teknologsektionen är fackligt, partipolitiskt och religiöst oberoende.
\subsection{Medlemmar}
Medlem i teknologsektionen är den som är inskriven vid utbildningsprogrammet Datateknik vid Chalmers och som erlagt sektionsavgift. Medlem är även före detta studerande vid Datateknik, Chalmers efter erlagd administrativ avgift. Därutöver kan teknologsektionen ha hedersmedlemmar.
\subsection{Verksamhetsår}
Teknologsektionens verksamhetsår löper från och med den första maj. Ordinarie mandatperiod är 1:a maj - 30:e april.
\subsection{Räkenskapsår}
Teknologsektionens och dess kommittéers räkenskapsår löper från 1:a maj till 30:e april.
\newpage

%Medlemmar%%%%%%%%%%%%%%%%%%%%%%%%%%%%%%%%%%%%%%%%%%%%%%%%%%%%%%%%%
\section{Medlemmar}
\subsection{Rättigheter}
\subsubsection{}
Varje medlem har närvaro-, yttrande-, förslags-, och rösträtt på sektionsmöte.
\subsubsection{}
Endast medlem är valbar till förtroendepost inom teknologsektionen. Revisorerna är undantagna föregående regel. Förtroendepost innebär vald av sektionsmötet eller sektionsstyrelsen.
\subsubsection{}
Medlem har rätt att ta del av mötesprotokoll och teknologsektionens övriga handlingar.
\subsection{Skyldigheter}
Medlem är skyldig att rätta sig efter teknologsektionens bestämmelser.
\subsection{Hedersmedlems rättigheter}
Hedersmedlem har närvaro- och yttranderätt på sektionsmöte.
\newpage

%Organisation%%%%%%%%%%%%%%%%%%%%%%%%%%%%%%%%%%%%%%%%%%%%%%%%%%%%%%%%%
\section{Organisation}
\subsection{Verksamhetsutövande}
Teknologsektionens verksamhet utövas på sätt denna stadga med tillhörande reglemente och ekonomiskt reglemente föreskriver genom:
\begin{enumerate}
\item Sektionsmötet
\item Sektionsstyrelse
\item Teknologsektionens valberedning
\item Teknologsektionens revisorer
\item Datatekniks Nämnd för Studier, DNS
\item Sektionskommittéer
\item Hobbykommittéer
\end{enumerate}
\subsection{Ansvarsförhållanden}
\subsubsection{}
Sektionsmötet är teknologsektionens högsta beslutande organ. Sektionsstyrelsen är sektionsmötets ställföreträdare.
\subsubsection{}
Sektionsmötet har till sitt förfogande valberedning, revisorer, sektionskommittéer, hobbykommittéer, studienämnd och sektionsstyrelsen.
\newpage

%Sektionsmötet%%%%%%%%%%%%%%%%%%%%%%%%%%%%%%%%%%%%%%%%%%%%%%%%%%%%%%%%%
\section{Sektionsmötet}
\subsection{Befogenheter}
Sektionsmötet är teknologsektionens högsta beslutande organ.
\subsection{Sammanträden}
Det skall hållas fyra ordinarie sektionsmöten, ett per läsperiod. Utöver detta kan extra sektionsmöten hållas.
\subsection{Utlysande}
\subsubsection{}
Sektionsmötet sammanträder på kallelse av talhenspresidiet eller av sektionsstyrelsen.
\subsubsection{}
Rätt att hos sektionsstyrelsen eller talhenspresidiet begära utlysande av sektionsmöte tillkommer ledamot i sektionsstyrelsen, inspektor, kårens inspektor, kårstyrelsen, teknologsektionens revisorer eller minst 25 av teknologsektionens medlemmar. Sådant möte ska hållas inom tio läsdagar.
\subsubsection{}
\label{sec:sektionsmote_utlysande}
Sektionsmöte skall utlysas minst fem läsdagar i förväg genom att kallelse enligt reglemente anslås. Inkomna motioner och propositioner skall anslås minst tre läsdagar i förväg.
\subsection{Åligganden}
\subsubsection{}
Senast dagen före ordinarie mandatperiods början skall följande behandlas på sektionsmöte:
\begin{itemize}
\item Omfördelning av sektionens och dess organs tillgångar.
\item Val av sektionsstyrelse.
\item Val av revisorer.
\item Val av inspektor om så är aktuellt.
\end{itemize}
\subsubsection{}
Senast dagen före verksamhetsårets början skall följande behandlas på sektionsmöte:
\begin{itemize}
\item Sektionsavgift för de två kommande terminerna.
\item Fastställande av preliminär budget för nästkommande verksamhetsår.
\end{itemize}
\subsubsection{}
Senast sex månader efter verksamhetsårets början skall följande behandlas på sektionsmöte:
\begin{itemize}
\item Sektionens och sektionskommittéernas års- och revisionsberättelse för föregående verksamhetsår.
\item Beslut om ansvarsfrihet.
\item Fastställande av budget för innevarande verksamhetsår.
\end{itemize}
\subsection{Beslutförhet}
\subsubsection{}
Sektionsmötet är beslutsmässigt om mötet är behörigt utlyst enligt stadgans kapitel~\ref{sec:sektionsmote_utlysande}.
\subsubsection{}
Om färre än 40 medlemmar är närvarande då beslut ska fattas, kan detta endast ske om ingen yrkar på bordläggning. Detsamma gäller beslut i frågor som ej har varit anslagna tre läsdagar i förväg.
\subsection{Motion}
Medlem som önskar ta upp fråga på föredragningslistan skall anmäla detta skriftligen till sektionsstyrelsen senast fem läsdagar före sektionsmöte.
\subsection{Överklagande}
Beslut av sektionsmötet som strider mot kårens eller sektionens stadga, reglemente, ekonomiska reglemente eller policy får undanröjas av kårfullmäktige. Sådant beslut ska tas upp till prövning om det begärs av en kårmedlem då det rör kårens stadga, eller sektionsmedlem då det rör teknologsektionens stadga, reglemente, ekonomiska reglemente eller policy.
\subsection{Omröstning}
\subsubsection{}
Röstning med fullmakt får ej ske.
\subsubsection{}
Omröstning skall ske öppet, om ej sluten votering begärs.
\subsubsection{}
Vid lika röstutfall äger mötesordförande utslagsröst, utom vid personval då lotten avgör.
\subsubsection{}
Då flera förslag ställs mot varandra skall röstningsförfarandet fastslås innan omröstning påbörjas.
\subsubsection{}
Alla frågor som behandlas på sektionsmötet avgörs med enkel röstövervikt om inget annat anges i stadgan. Nedlagda röster räknas ej.
\subsection{Närvaro- och yttranderätt}
Närvaro- och yttranderätt tillkommer medlem, hedersmedlem, kårledningsledamöter, inspektor, kårens inspektor, revisorer samt av mötet adjungerade icke-medlemmar.
\subsection{Förslagsrätt}
Förslagsrätt tillkommer medlem, inspektor samt av mötet adjungerade icke-medlemmar.
\subsection{Rösträtt}
Rösträtt tillkommer medlem.
\subsection{Protokoll}
Sektionsmötesprotokoll skall justeras av två av mötet valda justeringsmän. Justerat protokoll ska anslås senast tio läsdagar efter mötet.
\newpage

%Valberedning%%%%%%%%%%%%%%%%%%%%%%%%%%%%%%%%%%%%%%%%%%%%%%%%%%%%%%%%%
\section{Valberedning}
\subsection{Sammansättning}
\subsubsection{}
Sammankallande utses av sektionsstyrelsen.
\subsubsection{}
Representanter i valberedningen fastställs i reglementet.
\subsection{Ansvar}
Valberedningen ansvarar för samtliga nomineringar till förtroendeposter på teknologsektionen.
\subsection{Anslag}
Valberedningens nomineringar skall anslås minst fem läsdagar före sektionsmöte.
\subsection{Fri nominering}
Fri nominering är tillåten till alla poster utom till sektionsstyrelsens ordförande och kassör. Nomineringsbara till dessa poster är endast de som minst 24 timmar innan sektionsmöte då val ska ske anmält sitt intresse till sektionsstyrelsen.
\newpage

%Sektionsstyrelsen%%%%%%%%%%%%%%%%%%%%%%%%%%%%%%%%%%%%%%%%%%%%%%%%%%%%%
\section{Sektionsstyrelsen}
\subsection{Befogenheter}
Sektionsstyrelsen handhar i överensstämmelse med denna stadga, befintligt reglemente, befintligt ekonomiskt reglemente samt av sektionsmötet fattade beslut den verkställande ledningen av sektionens verksamhet.
\subsection{Sammansättning}
Sektionsstyrelsen består av:
\begin{itemize}
\item Ordförande
\item Vice ordförande
\item Kassör
\item Sekreterare
\item SAMO
\item i reglementet fastställda medlemmar
\end{itemize}
Ordförande och kassör i sektionsstyrelsen skall vara myndiga.
\subsection{Rättigheter}
Sektionsstyrelsen äger rätt att i namn och emblem använda teknologsektionens namn och dess symboler i enlighet med Chalmers Studentkårs policies.
\subsection{Ansvarighet}
Sektionsstyrelsen ansvarar inför sektionsmötet för teknologsektionens verksamhet och ekonomi.
\subsection{Firmateckning}
Ordförande i sektionsstyrelsen samt dess kassör tecknar teknologsektionens firma var för sig.
\subsection{Styrelsemöte}
Sektionsstyrelsen sammanträder minst tre gånger per läsperiod.
\subsection{Utlysande}
\subsubsection{}
Sektionsstyrelsen sammanträder på kallelse av ordförande eller vice ordförande i sektionsstyrelsen.
\subsubsection{}
Medlem av sektionsstyrelsen äger rätt att hos vice ordförande i sektionsstyrelsen begära utlysande av styrelsemöte. Sådant möte skall hållas inom 5 läsdagar.
\subsection{Beslutförhet}
Sektionsstyrelsen är beslutsmässigt när minst 50\% av medlemmarna är närvarande. Ordförande eller vice ordförande skall närvara.
\subsection{Överklagande}
Beslut av sektionsstyrelsen som strider mot kårens eller teknologsektionens stadga, reglemente, ekonomiska reglemente samt policy får undanröjas av kårens fullmäktige. Sådant beslut skall tas upp till prövning om det begärs av en kårmedlem då det rör kårens stadga, eller av teknologsektionsmedlem då det rör sektionens stadga, reglemente, ekonomiska reglemente eller policy.
\subsection{Protokoll}
Protokoll skall föras vid styrelsemöte, justeras av två medlemmar av sektionsstyrelsen och anslås på teknologsektionens anslagstavla senast fem läsdagar efter mötet.
\subsection{Avsättning}
\subsubsection{}
För att avsätta sektionsstyrelsen krävs att ärendet är anslaget senast tre läsdagar innan sektionsmöte, och minst 2/3 av de röstberättigade vid mötet är om beslutet ense.
\subsubsection{}
Vid detta möte skall interimsstyrelse och ny valberedning väljas. Interimsstyrelsen utfärdar kallelse till extra sektionsmöte där ny ordinarie styrelse skall väljas. Detta sektionsmöte skall hållas inom 15 läsdagar från det sektionsmöte då interimsstyrelsen valdes och under ordinarie terminstid.
\subsubsection{}
Interimsstyrelsen övertar ordinarie sektionsstyrelsens befogenheter och skyldigheter tills dess en ny ordinarie sektionsstyrelse är vald.
\newpage

%Studienämnden%%%%%%%%%%%%%%%%%%%%%%%%%%%%%%%%%%%%%%%%%%%%%%%%%%%%%%%%%
\section{Datatekniks Nämnd för Studier}
\subsection{Uppgift}
\subsubsection{}
Datatekniks Nämnd för Studier, DNS, har till uppgift att inom teknologsektionen övervaka studiefrågor.
\subsubsection{}
DNS består av ordförande och vice ordförande som båda väljs på sektionsmötet. Utöver detta kan DNS ha sekundanter som väljs av sektionsstyrelsen på DNS förslag för ett av sektionsstyrelsen specificerat uppdrag och under en av sektionsstyrelsen specificerad mandatperiod. Sekundanterna räknas inte som medlemmar vad gäller det ekonomiska reglementet.
\subsubsection{}
DNS ordförandes och vice ordförandes mandatperiod är densamma som verksamhetsåret. DNS sekundanters mandatperiod är den mandatperiod de blev invalda på.
\subsubsection{}
Kursutvärdering: DNS ansvarar för att följa upp och delta i utformningen och genomförande av kursutvärderingar.
\subsection{Rättigheter}
Datatekniks Nämnd för Studier äger rätt att i namn och emblem använda teknologsektionens namn och dess symboler i enlighet med Chalmers Studentkårs policies.
\subsection{Skyldigheter}
Datatekniks Nämnd för Studier är skyldig att rätta sig efter teknologsektionens bestämmelser och fattade beslut.
\newpage

%sektionskommittéer%%%%%%%%%%%%%%%%%%%%%%%%%%%%%%%%%%%%%%%%%%%%%%%%%%%%
\section{Sektionskommittéer}
\subsection{Definition}
\subsubsection{}
Sektionskommitté skall ha ordförande, kassör samt ett i reglemente fastställt antal förtroendeposter.
\subsubsection{}
Posterna tillsätts av sektionsmöte på förslag av teknologsektionens valberedning.
\subsubsection{}
Sektionskommittéerna skall verka för teknologsektionens bästa och ha en i reglemente fastslagen uppgift.
\subsubsection{}
Ordförande och kassör i sektionskommitté skall vara myndiga.
\subsection{Rättigheter}
Sektionskommitté äger rätt att i namn och emblem använda teknologsektionens namn och dess symboler i enlighet med Chalmers Studentkårs policies.
\subsection{Skyldigheter}
Sektionskommitté är skyldig att rätta sig efter teknologsektionens stadga, reglemente, ekonomiska reglemente och beslut.
\subsection{Ekonomi}
\subsubsection{}
Sektionskommittéernas verksamhet och ekonomi granskas av teknologsektionens revisorer.
\subsection{Förteckning}
Teknologsektionens sektionskommittéer är de i reglemente förtecknade.
\newpage

%Hobbykommittéer%%%%%%%%%%%%%%%%%%%%%%%%%%%%%%%%%%%%%%%%%%%%%%%%%%%%
\section{Hobbykommittéer}
\subsection{Definition}
\subsubsection{}
Kommittén skall ha ordförande och består i övrigt av intresserade medlemmar.
\subsubsection{}
Ordförande väljs enligt reglemente.
\subsubsection{}
Kommittéerna skall verka för teknologsektionens bästa och ha en i reglemente fastslagen uppgift.
\subsubsection{}
Ordförande i kommitté skall vara myndig.
\subsection{Rättigheter}
Kommitté äger rätt att i namn och emblem använda teknologsektionens namn och dess symboler i enlighet med Chalmers Studentkårs policies.
\subsection{Skyldigheter}
Kommitté är skyldig att rätta sig efter teknologsektionens stadga, reglemente, ekonomiska reglemente och beslut.
\subsection{Förteckning}
Teknologsektionens hobbykommittéer är de i reglemente förtecknade.
\newpage

%Hedersmedlemmar%%%%%%%%%%%%%%%%%%%%%%%%%%%%%%%%%%%%%%%%%%%%%%%%%%%%%%%
\section{Hedersmedlemmar}
\subsection{Grundkrav}
Till hedersmedlem kan kallas person som synnerligen främjat sektionens intressen och strävande.
\subsection{Förslag och kallande}
Förslag till hedersmedlem lämnas i skrivelse till sektionsstyrelsen senast fem läsdagar innan sektionsmötet undertecknad av minst 25 av sektionens medlemmar. Beslut om kallande av hedersmedlem fattas vid nästkommande sektionsmöte och är enbart giltigt om det antas med två tredjedelar av antalet röster. Antager kallad person kallelsen är han/hon officiellt hedersmedlem.
\newpage

%Skyddshelgon och sektionsfärger%%%%%%%%%%%%%%%%%%%%%%%%%%%%%%%%%%%%%%%
\section{Skyddshelgon och sektionsfärger}
\subsection{Skyddshelgon}
Teknologsektionens skyddshelgon är Hacke Hackspett.
\subsection{Sektionsfärg}
Teknologsektionens färg är orange.
\newpage

%Protokoll och anslagning%%%%%%%%%%%%%%%%%%%%%%%%%%%%%%%%%%%%%%%%%%%%%%
\section{Protokoll och anslagning}
\subsection{Allmänt}
Protokoll som föres i teknologsektionens olika organ skall innehålla anteckningar om ärendenas art, samtliga ställda och ej återtagna yrkanden, beslut samt särskilda yttranden och reservationer.
\subsection{Anslagning}
Meddelanden och beslut är behörigt anslagna då de anslås på teknologsektionens officiella anslagstavla.
\newpage

%Revision och ansvarsfrihet%%%%%%%%%%%%%%%%%%%%%%%%%%%%%%%%%%%%%%%%%%%%
\section{Revision och ansvarsfrihet}
\subsection{Revisorer}
\subsubsection{}
Sektionsmötet utser 2–4 lekmannarevisorer med uppgift att granska teknologsektionens verksamhet och ekonomi under verksamhetsåret.
\subsubsection{}
Teknologsektionens revisorer kan ej inneha annan förtroendepost på teknologsektionen under sitt verksamhetsår.
\subsubsection{}
Räkenskaper och övriga handlingar skall tillställas revisorerna senast 15 läsdagar före sektionsmöte.
\subsection{Åligganden}
\subsubsection{}
Det åligger revisorerna att på teknologsektionens officiella anslagstavla anslå revisionsberättelser senast tre läsdagar före ordinarie sektionsmöte.
\subsubsection{}
Revisionsberättelsen skall innehålla yttrande ifråga om ansvarsfrihet för berörda personer.
\subsection{Ansvarsfrihet}
\subsubsection{}
Ansvarsfrihet är beviljad berörda personer då sektionsmötet fattat beslut om detta.
\subsubsection{}
Skulle förtroendevald på teknologsektionen med ekonomiskt ansvar avgå före mandatperiodens slut, skall revision företagas.
\newpage

%Avgifter%%%%%%%%%%%%%%%%%%%%%%%%%%%%%%%%%%%%%%%%%%%%%%%%%%%%%%%%%%%%%%
\section{Avgifter}
\subsection{Sektionsavgift}
Varje studerandemedlem av teknologsektionen skall erlägga beslutad sektionsavgift.
\subsection{Administrationsavgift}
Ej studerandemedlem skall erlägga en beslutad administrationsavgift.
\newpage

%Teknologsektionens upplösning%%%%%%%%%%%%%%%%%%%%%%%%%%%%%%%%%%%%%%%%%
\section{Teknologsektionens upplösning}
\subsection{Beslut om upplösning}
Teknologsektionen upplöses genom beslut på två på varandra följande sektionsmöten, med minst femton läsdagars mellanrum, med minst 60 eller samtliga medlemmar närvarande. För att beslutet skall vara giltigt krävs att det antas med tre fjärdedelar av antalet röster.
\subsection{Tillgångar och nystart}
Om sektionsmötet beslutar att upplösa teknologsektionen skall samtliga dess tillgångar och skulder, som framgår av upprättad balansräkning, i och med upplösningen tillfalla Chalmers studentkår att förvalta tills dess att en ny förening eller teknologsektion bildas som representerar studerande på utbildningsprogrammet för Datateknik, Chalmers.
\newpage

%Ändrings- och tolkningsfrågor%%%%%%%%%%%%%%%%%%%%%%%%%%%%%%%%%%%%%%%%%
\section{Ändrings- och tolkningsfrågor}
\subsection{Stadgeändringar}
\subsubsection{}
Ändring av denna stadga kan endast göras av sektionsmötet. För att vara giltig måste ändringen antas med två tredjedelar av antalet röster vid två på varandra följande sektionsmöten, varav minst ett ordinarie, med minst tio läsdagars mellanrum.
\subsubsection{}
Ändring av eller tillägg till denna stadga skall godkännas av kårstyrelsen.
\subsection{Reglementesändring}
Ändring av eller tillägg till reglementet eller det ekonomiska reglementet kan endast göras av sektionsmötet. För att vara giltig måste ändringen antas med två tredjedelar av antalet röster.
\subsection{Tolkningstvist}
\subsubsection{}
Uppstår tolkningstvist om dessa stadgars tolkning, tolkas stadgan av inspektor för avgörande. Om sådan ej finns avgörs frågan av Chalmers studentkårs inspektor.
\subsubsection{}
Vid tolkning av reglemente eller ekonomiskt reglemente gäller, tills frågan avgjorts av sektionsmötet, sektionsstyrelsens tolkning.
\newpage

%Inspektor%%%%%%%%%%%%%%%%%%%%%%%%%%%%%%%%%%%%%%%%%%%%%%%%%%%%%%%%%%%%%
\section{Inspektor}
\subsection{Allmänt}
Inspektor skall ägna uppmärksamhet åt och stödja teknologsektionens verksamhet. Inspektor skall därvid hållas underrättad om teknologsektionens verksamhet. Inspektor har rätt att ta del av teknologsektionens protokoll och övriga handlingar.
\subsection{Val}
Inspektor väljs av sektionsmötet för en tid av två kalenderår.
\newpage

%Hedersbetygelser%%%%%%%%%%%%%%%%%%%%%%%%%%%%%%%%%%%%%%%%%%%%%%%%%%%%%%
\section{Hedersbetygelser}
\subsection{Allmänt}
Teknologsektionen kan som tack eller hedersbetygelse utdela barspeglar.
\subsection{Kriterier}
För att mottagare skall anses värdig att mottaga en barspegel bör något av nedanstående kriterium vara uppfyllda:
\begin{itemize}
\item ha gjort sektionen en betydande tjänst
\item ha gjort sektionen en betydande björntjänst
\item vara monark och fylla jämt
\end{itemize}
\subsection{Införskaffande}
Det åligger sektionsstyrelsen att tillse att erforderlig mängd barspeglar finnes.


\end{document}

% LaTeX-inställningar%%%%%%%%%%%%%%%%%%%%%%%%%%%%%%%%%%%%%%%%%%%%%%%%%%
% Kräver dtek-tex (github.com/dtekcth/dtek-tex)
% KOMPILERAS MED xelatex!
\documentclass[a4paper]{dtek}
\setcounter{secnumdepth}{5}
\title{Constitution}

\newcommand{\sdocsae}{Student Division of Computer Science and Engineering~}

%FYLL I VID ÄNDRINGAR%%%%%%%%%%%%%%%%%%%%%%%%%%%%%%%%%%%%%%%%%%%%%%%%%%
\newcommand{\updated}{2019-02-07} %Insert

%Dokumentstart%%%%%%%%%%%%%%%%%%%%%%%%%%%%%%%%%%%%%%%%%%%%%%%%%%%%%%%%%
\begin{document}
\makeheadfoot

%Titlepage%%%%%%%%%%%%%%%%%%%%%%%%%%%%%%%%%%%%%%%%%%%%%%%%%%%%%%%%%%%%%
\vspace*{\fill}
\begin{center}
{\Huge \textbf{Financial Regulations of the \sdocsae}}
\par\bigskip
\includegraphics[width=300pt]{dteklogo.pdf}
\par\bigskip
{\LARGE Chalmers, Gothenburg}
\end{center}
\vspace*{\fill}
\begin{center}
{\LARGE Updated: \updated}
\end{center}
\vspace*{\fill}


\newpage
\setcounter{tocdepth}{1}
\tableofcontents
\newpage


%Innan vi börjar...%%%%%%%%%%%%%%%%%%%%%%%%%%%%%%%%%%%%%%%%%%%%%
\setcounter{section}{-1}
\section{Before we begin\dots}
This is a translation of the Swedish version of the document. In case there's a difference between the English and Swedish versions, the Swedish version prevails.

This translation was written by Tove and Hugo, and is guided by the dictionary written by Talhenspresidiet/Speakers' Council '18

The English terms for meeting formalities in this document are mostly taken from ``Robert's Rules of Order, Newly Revised'', a book which includes the rules most commonly used in the U.S. when conducting meetings.

\newpage

%Allmänt%%%%%%%%%%%%%%%%%%%%%%%%%%%%%%%%%%%%%%%%%%%%%%%%%%%%%%%%%%%%%%%
\section{General}
\subsection{}

These regulations were created to clarify what the division's funds may and may not be used for. It is therefore meant as an aid to the board, lay-auditors and committees in their work. The basic philosophy is that the division's money is meant for all its members, which in practice means that events that are not open to all members of the division are not allowed to burden the division's finances. It is important to remember when interpreting the financial regulations that they were not created to needlessly hinder any committee's ability or lust for fun events and escapades. 

\subsection{Price base amount}
\begin{itemize}
  \item The price base amount that is used during the entire fiscal year is the price base amount that is current at the start of that fiscal year. 

  \item Partial price base amounts are rounded up to the nearest hundred kronor. 
\end{itemize}

%D-Styret%%%%%%%%%%%%%%%%%%%%%%%%%%%%%%%%%%%%%%%%%%%%%%%%%%%%%%%%%%%%%%
\section{The board of the division}
\subsection{}
The board of the division is responsible for the entirety of the division's finances. 

\subsection{}
The chairman of the board must
\begin{itemize}
  \item Sign the division %oklart
  \item Ensure that the chairman of each committee has access to and knowledge of the division's constitution, by-laws and financial regulations
\end{itemize}
\subsection{}
The treasurer of the board must
\begin{itemize}
  \item Sign the division%oklart
  \item Ensure that the treasurer of each committee has access to and knowledge of the division's constitution, by-laws and financial regulations
  \item Continuously monitor the division's bookkeeping
  \item In consultation with the board present a budget proposal for the first ordinary autumn division meeting
  \item At each division meeting be able to report the financial status of the division
  \item Educate new trustees within the division in how the division's bookkeeping and audit systems should be used   
  \item Be responsible for the board's petty cash
\end{itemize}

\subsection{}
The board of the division has the right to within the budget of the board represent the division and arrange food for its members in connection with its meetings. That food shall be a post in the budget for the board.


%Sektionsföreningar%%%%%%%%%%%%%%%%%%%%%%%%%%%%%%%%%%%%%%%%%%%%%%%%%%%%
\section{Division committees}
\label{sec:sektionsforeningar}
\subsection{}
The chairman of a division committee is obliged to:
\begin{itemize}
    \item Continuously keep the board of the division updated as to the committee's economic situation.
    \item Together with the treasurer of the committee make sure that the committee manages its assets in accordance with the division's constitution, by-laws and other decisions.
    \item Together with the treasurer of the committee sign a binding contract with the board of the division about guilt if there's faulty bookkeeping.
    \item Together with the treasurer of the committee sign the committee.%oklart
\end{itemize}

\subsection{}
The treasurer of a division committee is obliged to:
\begin{itemize}
    \item Bookkeep in a way that is approved by the division's lay-auditors.
    \item Together with the chairman of the committee make sure that the committee manages its assets in accordance with the division's constitution, by-laws and other decisions.
    \item Together with the chairman of the committee sign a binding contract with the board of the division about guilt if there's faulty bookkeeping.
    \item Present the economic situation of the committee on each division meeting.
    \item Archive the division's bookkeeping, on a place decided by the board of the division, during the time required by law for the type of organization that division is.
    \item Together with the chairman of the committee sign the committee.%oklart
\end{itemize}

\subsection{}
Every committee shall start a new cashbook in the start of their term. %MYCKET oklart

\subsection{Requests}
Committees that wish for extra economic means for their operations shall send a request and their reasons for it to the board of the division. The board can approve extra means if the amount requested does not exceed 0.25 price base amounts. If the amount is higher the division meeting must approve.

\subsection{Capital}
\label{sec:sektionsforeningar_startkapital}
\subsubsection{Starting capital}
DRust, DAG, Delta, D6 and DNollK shall at the start of their term have 0 price base amounts in assets. DRust, Delta, D6 and DNollK has the right to borrow up to 0.25 price base amounts from the board of the division. DAG has the right to borrow up to 1.25 price base amounts.

\subsubsection{Ending capital}
The assets of DRust, DAG, Delta, D6 and DNollK and their debts shall be passed on to the board of the division at the end of their terms.

\subsubsection{Support capital}
The board of the division has the right to temporarily lend money to a committee to pay urgent debts when waiting for incoming capital.

\subsection{Budget}
For events with an expected revenue exceeding 0.75 price base amounts a thorough budget shall be made and approved by the board of the division before the event may be carried out.

\subsection{Work clothing}
A committee may spend up to 0.025 price base amounts per person on work clothing. Reasonable work clothing can be considered to be an overall or equivalent, t-shirts and other shirts.

\subsection{Workers' food}
When arranging in the operations of a committee their economy may be charged for expenditures for purchasing workers' food depending on the time spent on the event:
\begin{itemize}
    \item 1--4 hours: 30 kr/per person (approx. 0.00075 price base amounts)
    \item 4--8 hours: 60 kr/per person (approx. 0.0015 price base amounts)
    \item More than 8 hours: 90 kr/per person (approx. 0.00225 price base amounts)
\end{itemize}
These expenditures may not exceed the economic capacity of the committee. Cooking the workers' food may not be counted in the time spent on the event.

%Intresseföreningar och andra grupper under sektionen%%%%%%%%%%%%%%%%%%
\section{Division clubs and other groups under the division}
\subsection{}
Economic capital for division clubs and other groups under the division shall be regulated by the budget for the board of the division.
\subsection{}
In the case that a division club or other group under the division has been given their own economy, §\ref{sec:sektionsforeningar} with the exception of §\ref{sec:sektionsforeningar_startkapital} also apply to those.

%Internrepresentation%%%%%%%%%%%%%%%%%%%%%%%%%%%%%%%%%%%%%%%%%%%%%%%%%%
\section{Internal representation}
\subsection{}
\label{sec:internreps}
Every division committee and DNS has the right to burden their economy with the following: 
\begin{itemize}
    \item One or more team building activities for a total of at most 0.015 price base amounts per person.
    \item Transfer for at most 0.025 price base amounts.
    \item Asping and asp-clothes for a total of at most 0.065 price base amounts.
\end{itemize}
The above may not go above the committee's economic assets. Alcoholic drinks may not be funded with the committee's funds at the above events.

\subsection{}
The board of the division may give division clubs or other groups under the division rights corresponding to §\ref{sec:internreps} if it's motivated.

%Sponsring%%%%%%%%%%%%%%%%%%%%%%%%%%%%%%%%%%%%%%%%%%%%%%%%%%%%%%%%%%%%%
\section{Sponsors}
\subsection{}
Committees may not search for sponsors without first speaking with the division's business relations committee, DAG.

%Fonder%%%%%%%%%%%%%%%%%%%%%%%%%%%%%%%%%%%%%%%%%%%%%%%%%%%%%%%%%%%%%%%%
\section{Funds}
\subsection{Capital fund}
\subsubsection{Purpose}
\label{sec:kapitalfond_syfte}
The purpose of the capital fund is to not burden the division or its committees with big investments. The funds shall be used for things that have a permanent value, and are useful for the members of the division directly or indirectly. Funds may not be used for things the premise fund is meant for.

\subsubsection{Management}
The fund is managed by the board of the division.

\subsubsection{Setting aside}
\begin{itemize}
    \item An amount of money specified in the budget, and approved by the division meeting, shall be set aside for the capital fund.
    \item All returns on the fund shall be returned to the fund.
\end{itemize}
\subsubsection{Withdrawal}
\begin{itemize}
    \item The board of the division has the right to approve withdrawals from the fund not exceeding 0.25 price base amounts per year. Withdrawals of amounts exceeding 0.25 price base amounts has to be approved by the division meeting before funds are paid. Every withdrawal shall be reported to the next division meeting.
    \item The division meeting has the right to, if there's a good reason, withdraw funds contrary to the purpose in §\ref{sec:kapitalfond_syfte}.
\end{itemize}
\subsubsection{Auditing}
Auditing shall be undertaken each year by the division's lay-auditors. Auditing coincides with other audits of the division.

\subsection{Premise fund}
\subsubsection{Purpose}
\label{sec:lokalfond_syfte}
The purpose of the premise fund is to secure funds for maintenance and repairs of the division's premises, and to purchase inventory. The money shall be used for bigger repairs and repaintings of the division premises as well as furniture for premises where all division members have access.
\subsubsection{Management}
The fund is managed by the board of the division.
\subsubsection{Setting aside}
\label{sec:lokalfond_avsattning}
\begin{itemize}
    \item Ten (10) \% of the division fees during a year is deposited into the premise fund.
    \item An amount of money specified in the budget, and approved by the division meeting, shall be set aside for the premise fund.
    \item All returns on the fund shall be returned to the fund.
\end{itemize}
\subsubsection{Withdrawals}
\begin{itemize}
    \item The board of the division has at its disposal 3/4 of the capital deposited that year, in accordance with the first point in §\ref{sec:lokalfond_avsattning} for basic maintenance of the division premises. The board of the division has the right to decide about withdrawals over that amount, however that must be reported to the next division meeting. At that meeting completed or planned purchases, repairs and maintenance must also be reported.
    \item The division meeting has the right to, if there's a good reason, withdraw funds contrary to the purpose in §\ref{sec:lokalfond_syfte}.
\end{itemize}
\subsubsection{Auditing}
Auditing shall be undertaken each year by the division's lay-auditors. Auditing coincides with other audits of the division.
\end{document}
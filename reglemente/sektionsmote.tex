\section{Sektionsmötet}

\subsection{Kallelse} 
Kallelse till sektionsmöte består av förslag till dagordning. Förslag till dagordning skall anslås på teknologsektionens officiella anslagstavla.

Teknologsektionens officiella anslagstavla skall vara placerad i Basen. 

\subsubsection{Förslag till dagordning}
Förslag till dagordningen skall innehålla:
\begin{itemize}  
  \item Datum, tid och plats för mötet 
  \item Mötets öppnande 
  \item Val av mötesordförande 
  \item Val av mötessekreterare 
  \item Val av två justeringsmän tillika rösträknare 
  \item Föregående mötes protokoll. 
  \item Mötets behöriga utlysande 
  \item Fastställande av mötesordning 
  \item Adjungeringar 
  \item Meddelanden 
  \item Eventuella verksamhetsrapporter från sektionskommittéer 
  \item Eventuella års- och revisionsberättelser 
  \item Eventuella personval 
  \item Propositioner 
  \item Motioner 
  \item Övriga frågor 
\end{itemize}
\subsection{Mötesplats} 
Sektionsmötet skall hållas på Chalmersområdet. 
%\subsection{Mötesordning}
%\subsubsection{Begärande av ordet}
%Ordet begärs genom handuppräckning och delas i tur och ordning ut av mötesordförande. 
%\subsubsection{Replik}
%Om ett anförande berör en speciell person, har denne rätt till replik om högst en minut. Replik skall hållas i direkt anslutning till anförandet. En kontrareplik om högst en minut kan beviljas. Kontra-kontra-replik beviljas ej. 
%\subsubsection{Ordningsfråga}
%Debatt i ordningsfråga bryter debatt i sakfråga och skall avgöras innan debatt i sakfråga fortsätter. 
%\subsubsection{Streck i debatten}
%Streck i debatten behandlas som ordningsfråga. Bifalls frågan om streck i debatten skall mötesordföranden justera talarlista. Därefter får endast de som står på listan yttra sig i frågan och inga nya yrkanden i sakfrågan får framställas. 
%Upphävande av streck i debatten behandlas även det som ordningsfråga. 
%\subsubsection{Yrkande}
%Yrkande framställs både muntligt och skriftligt till mötesordförande. 
%\subsubsection{Reservation}
%Reservation mot beslut av sektionsmötet skall anmälas muntligt i omedelbar anslutning till beslutet och skriftligen senast 24 timmar efter mötet. 
%\subsubsection{Ajournering}
%Behandlas som ordningsfråga. Bifalls yrkandet om ajournering skall tidslängden av ajourneringen fastställas. 
\subsection{Motion} 
Motion som är upptagen på dagordningen måste lyftas och föredragas av motionären eller någon annan på mötet med förslagsrätt, annars faller motionen. Vidare skall styrelsen lämna sitt utlåtande om motionen och därefter följer allmän debatt. 
\subsection{Talhenspresidiet}
\subsubsection{Medlemmar}
Talhenspresidiet består av talhen, vice talhen och sekreterare.
\subsubsection{Uppdrag}
Det åligger talhenspresidiet att utlysa och kalla till ordinarie sektions-
möten i enlighet med stadgans 4.2. Talhen är sektionsmötets ordförande. Vice talhen träder vid talhens frånvaro in i dennes ställe. Sekreteraren är sektionsmötets mötessekreterare.
\subsubsection{Val}
Talhen, vice talhen och sekreterare väljs av sektionsmötet.
\subsubsection{Mandatperiod}
Mandatperioden är densamma som verksamhetsåret.
\subsubsection{Talhenspresidiets beslut}
Talhenspresidiets beslut tas av talhen. Denna beslutsmakt övergår till
vice talhen vid talhens förfall.

\newpage

